\section{Semantics}

\subsection{Implémentation choisie}

Le module \textit{Semantics} est découpé en plusieurs fonctions:
\begin{itemize}
\item[$\bullet$] {\textbf{matching\_expression} pour parser les \textit{expressions};}
\item[$\bullet$] {\textbf{matching\_predicate} pour parser les \textit{predicates};}
\item[$\bullet$] {\textbf{assign}, \textbf{keepargs} \& \textbf{keepargs\_except} pour la conservation des valeurs des variables;}
\item[$\bullet$] {\textbf{formula}, point d'entrée du module \textit{Semantics} qui se chargera des \textit{assignations}, des \textit{skips} et des \textit{guards}.}\\
\end{itemize}

L'implémentation de la division par 0 aura entraîné une refonte de l'architecture du module. En effet, celle-ci crée une restriction supplémentaire à prendre en compte: les opérations en dénominateur doivent être différentes de 0.\\
Pour ne pas perdre d'informations, les retours de fonctions ne seront plus de la forme \textit{Z3.Expr.expr} mais des tuples de la forme (\textit{Z3.Expr.expr}, \textit{Z3.Expr.expr list}). Le deuxième paramètre représentera les contraintes sur les divisions.

Aussi, concernant la division, on peut noter que les simplifications (comme le produit en croix pour les divisions) lors de la construction de la séquence d'opérations ne sont pas implémentées.

\subsection{Tests ajoutés}

Les tests ajoutés pour la partie \textit{Semantics} sont présents dans le fichier \textit{Test\_Semantics.ml}. Ceux-ci sont basés sur ceux déjà présents et ne font que tester les combinaisons d'\textit{Opérations}. Les tests sont effectués automatiquement à l'aide de la commande \textbf{make test} et, afin des les distinguer de ceux déjà fournis, leur nom est précédé du terme \textit{``supp\_''}.\\

Quelques tests implémentés pour les \textit{assign}:

\begin{equation*}
  \begin{aligned}
    z = \frac{x + 3}{y},\ z = \frac{x + 3}{\frac{y}{z - 4}}
  \end{aligned}
\end{equation*}

Ainsi que pour les \textit{guard}:

\begin{equation*}
  \begin{aligned}
    \frac{x + 5}{z} > \frac{y}{z},\ \frac{x - 5}{z} < \frac{y}{z + y}
  \end{aligned}
\end{equation*}

Bien évidemment, aucun nouveau test n'a été effectué pour la partie \textit{skip}.
